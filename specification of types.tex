\documentclass{article}
\usepackage[utf8]{inputenc}
\usepackage{Arvo}
\usepackage[T1]{fontenc}

\title{A Possible Formalization of Types Founded in Geometric Structure}
\author{Daniel Smith}
\date{March 2022}

\begin{document}

\maketitle

\section*{Introduction}
Types, as they are used in mathematics, provide a means to make generalized statements about mathematical expressions.
One can reason about a collection of expressions rather than being forced to observe each individual subject.
In this report, I outline a formalization of types which is founded in the principals of geometry.
The resulting system can extend logical reasoning by taking advantage of existing geometric relationships and properties.
Reasoning with types is of special significance to modelling and proving complex algorithms and functions.
I hope that this treatment of types may offer new insight about their behavior.

\section*{Definition of Type}

\begin{enumerate}
    \item A type can be considered a geometric structure with features.
    \item A feature can be any consideration such as dimension, size, location, or any other arbitrary thing.
    \item Every type has a countable number of features from none to infinite.
    \item Multiple types may have the same feature. A feature which is shared between multiple types is a common feature.
    \item There is a single type which has no features and it is the \textit{Null} type.
    \item There is a single type which has all possible features and it is the \textit{Any} type.
\end{enumerate}

\section*{Relationships Between Types}
\begin{enumerate}
    \item Consider two types which have the same number of features and each feature is common. These two types are \textit{equivalent}.
    \item Consider two types which do not have the same number of features and no feature is common. These two types are \textit{disjoint}.
    \item Consider two types which may have the same number of features and not all features are common. These two types are \textit{alike}.
    \item Consider two types which are alike and only one of the types has features which are not common. The type with the greater number of features is the \textit{super-type}. The type with the lesser number of features is the \textit{sub-type}.
\end{enumerate}

\section*{Type Constructions}
\begin{enumerate}
    \item A \textit{unit} is a type which has a single feature and consequently serves to identify the feature it possesses.
    \item A \textit{function} is a transformation from a collection of types to another collection of types.
    \item A \textit{composite} is a type which has a common feature of each and every feature in a collection of types.
    \item A \textit{type superposition} is a type which is possibly equivalent to any one type in a collection of types and cannot be equivalent to every type in that collection.
\end{enumerate}


\section*{Relation of Types with Semantics}
Types can be used to reason about mathematics by modelling the properties of various mathematical concepts.
Thus an associative mapping must be devised between a system of types and a system of mathematics.
Considering a given mathematical expression may have values, operations, variables, constructs, and other concepts, the following mappings are possible:
Unit types may be associated with values, function types with operation, variables with composites, and structures or other classes of entities with superposition.
Using these mappings, a typed model of mathematics can be constructed which has a one-to-many correspondence with a set of possible mathematical expressions which can be constructed from the model.
If the model is proven to be sound, then it's associated set of expressions must also be sound.

A simple example is to model the system of natural numbers.
The most natural way to achieve this is to take advantage of the countability of a type's features.
If the Null type is associated with zero, then each successive super-type of the Null type will possess at least one more feature and can be associated with increasing values.
i.e. if the Null type corresponds to zero, then the super-type of Null corresponds to one, the super-type of the super-type of Null corresponds to two, and so on.

An alternative extensional definition of the natural numbers is to associate a unique unit type with each natural number. The "set" of natural numbers is then the composite of all the unit types. However, this approach is more difficult to express without abbreviation since the set of natural numbers is infinite.

A more arbitrary example which is closer to the geometric nature of this system of types is the statement: 
\begin{center} 
The type, \textit{third-dimension}, is a \textbf{superposition} of every type \textit{alike} the \textbf{composite} of the \textbf{units} \textit{length}, \textit{width}, and \textit{height}.
\end{center}
This trivially models the relation that that class of three dimensional spaces must each feature length, width, and height (not accounting for other factors such as whether the space is closed, if it has folds or intersections, etc..).

\end{document}